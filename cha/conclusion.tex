\documentclass[../main]{subfiles}
\begin{document}

\chapter{总结}%
\label{cha:总结}

在军用、安防、道路检测和工业产品检测领域,红外成像系统的使用越来越广泛。其中,红
外焦平面阵列结构简单紧凑,又具有高的信噪比和灵敏度,因此是目前最具有发展潜力的红
外成像器件。

但是由于材料和制造技术等因素的限制,红外系统普遍存在非均匀性问题,即:焦平面探测
单元的响应率很难做到一致,这就会造成对着均匀辐射目标最后探测器给出的响应电压不一
样,带来的直接后果就是红外焦平面阵列成像的效果。这严重影响了红外系统的成像质量,
因此必须采取非均匀性校正算法对焦平面的像元响应进行校正。

表\ref{tab:时间线}按照时间顺序整理了具有重要意义的一部分相关研究成果。更早的参考
文献已经提出了大多数经典算法。所以列出了大多数近二十年来提出的新算法。更新的参考
文献引用量较少,暂不列出。

\begin{table}
	\centering
	\caption{时间线}
	\label{tab:时间线}
	\begin{tabular}
		{@{\,}r <{\hskip 2pt} !{\color{LightSteelBlue3}\makebox[0pt]{\textbullet}\hskip-0.5pt\vrule width 1pt\hspace{\labelsep}} >{\raggedright\arraybackslash}p{5cm}}
		1997 & Scribner 预言自适应技术在未来将非常有用。 \\
		2002 & Schulz 定义了估计校正后的剩余噪声的校正性能指标。 \\
		2004 & Strojnik 证明了常数统计法的收敛速度和计算复杂度更优。 \\
		2007 & Hardie 率先使用了总方差法。 \\
		2010 & Torres 不假设非均匀性的种类或数量下采用神经网络法。 \\
		2015 & Godoy 提出了噪声抵消法。 \\
		2016 & Hardie 改进了Scribner 的算法。 \\
		2016 & Yong 提出了帧间预测方法。 \\
		2017 & Nie 提出了快速独立分量盲分离法。 \\
		2018 & Esteban 提出了各向同性总变化法。 \\
		2019 & Tendero 提出了单帧去条纹法。 \\
	\end{tabular}
\end{table}

纵观国内外十年来基于场景校正算法的发展,可以预见今后基于场景校正的研究方向与发展
趋势:

\begin{enumerate}
	\item 新的校正方法与思路:利用更加丰富多彩的假设理论与数学工具;
	\item 更完善的理论模型:如探测器环境温度,多参量对像元的响应进行精确刻画;
	\item 更高的收敛速度:用十几帧甚至几帧就可以完成非均匀性校正;
	\item 更少的不良效果:进一步完善假设,减少鬼影效应;
	\item 更低的运算量与存储量:更加易于系统集成与硬件实时实现。
\end{enumerate}

\end{document}

